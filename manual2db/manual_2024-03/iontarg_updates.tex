In a \textbf{static} simulation, the target presents itself to all incoming 
ions as specified in the input file and, optionally, any file read during
setup (histogram, geometry, or cell file).

In a \textbf{dynamic} simulation, the target is updated according to the
changes caused by the simulated collision cascades. 
IMSIL supports two kinds of dynamic simulation modes:
First (``damage accumulation''), the damage to crystalline targets may be
taken into account for subsequent trajectories. 
In this way, the suppression of channeling by damage accumulation may be 
described. 
Also, recombination of newly generated defects with damage from previous 
collision cascades can be taken into account. 
Second (``dynamic simulation''), it is possible to predict the evolution of 
the target geometry from the changes in the atom densities of the target.
In the version described in this manual, only 1-D targets can be updated in 
this way.

The updates can be done continuously or with a certain frequency. 
In the sequential version of IMSIL, damage and atom densities are, by
default, updated as soon as they occur. 
In the parallel version this would lead to excessive communication between 
the threads. 
Therefore, the user has to choose the frequency of these updates in terms of 
the number of simulated ions between updates. 
Too frequent updates lead to penalties in execution time, while too infrequent 
updates may affect the results. 
In case of geometry updates, it is also possible to specify that updates be
performed only when cell densities have changed more than a threshold. 
When a geometry update is performed, also the grid for the atom densities is
adjusted so that the grid spacings remain approximately the same.

\begin{center}
\begin{tabular}{lll}
parameter \quad                       & IMSIL name & to be specified in record \\
\hline
histogram file                        & \texttt{HISFILE}  & \texttt{\&SETUP} \\
geometry file                         & \texttt{GEOMFILE} & \texttt{\&GEOMETRY}
\\
cell file                             & \texttt{CELLFILE} & \texttt{\&SETUP} \\
take damage accumulation into account & \texttt{LDAMDYN}  & \texttt{\&SETUP} \\
dynamic simulation mode               & \texttt{LDYN}     & \texttt{\&SETUP} \\
number of ions for histogram updates  & \texttt{NIONHIS}  & \texttt{\&SETUP} \\
number of ions for geometry updates   & \texttt{NIONGEOM} & \texttt{\&SETUP} \\
refined damage model                  & \texttt{LDAM}     & \texttt{\&DAMAGE} \\
threshold for geometry update         & \texttt{FRACUPD}  & \texttt{\&GEOMETRY} \\
fixed coordinate in 1-D geometry update & \texttt{POSFIX}  & \texttt{\&GEOMETRY} \\
\end{tabular}
\end{center}

