This record is used to define the parameters of the electronic stopping model.
For a description of the parameters refer to Chapter \ref{s:stop}. The
\texttt{\&SEPAR} record has the index variables \texttt{ATOM1} and
\texttt{ATOM2}. \texttt{ATOM1} corresponds to the index of the projectile, while
\texttt{ATOM2} corresponds to the index of the target atom/recoil. Regarding the
definition of atom indices, see Section~\ref{s:atom}.

\begin{keydescription}{\texttt{CORLIN} --- Correction to Lindhard stopping}
%
  This parameter specifies the correction factor $k_\mathrm{corr}$ in the modified 
  Lindhard stopping formula $S_e=k_\mathrm{corr} k_\mathrm{L} \sqrt{E}$.  
  \texttt{CORLIN} is only effective if \texttt{SEMODEL='Lindhard'}.
  \begin{keytab}
    Type:    \> index variable array (\texttt{ATOM2},\texttt{ATOM1}) 
                of real \\
    Default: \> see \cite{I0104} for Si targets \\
             \> 1.1 for \texttt{Z1=5} in SiO$_2$, \\
             \> 1 otherwise \\
    Range:   \> $\ge 0$
  \end{keytab}
\end{keydescription}

\begin{keydescription}{\texttt{C0BETHE} --- Bethe Bloch stopping power constant}
%
  This parameter specifies the Bethe constant $c_0$ to be applied in
  calculation of the Bethe-Bloch stopping power.
  \texttt{C0BETHE} is only effective if \texttt{SEMODEL='Lindhard'}.

  \begin{keytab}
    Type:    \> index variable array (\texttt{ATOM2},\texttt{ATOM1}) 
                of real \\
    Default: \> 1.43 for \texttt{Z1=1} \cite{Hob05a} \\
             \> 1 otherwise \\
    Range:   \> arbitrary
  \end{keytab}
\end{keydescription}
%

\begin{keydescription}{\texttt{C1BETHE} --- Bethe Bloch stopping power linear factor}
%
  This parameter specifies the linear Bethe factor $c_1$ to be applied in
  calculation of the Bethe-Bloch stopping power.
  \texttt{C1BETHE} is only effective if \texttt{SEMODEL='Lindhard'}.

  \begin{keytab}
    Type:    \> index variable array (\texttt{ATOM2},\texttt{ATOM1}) 
                of real \\
    Default: \> 40*\texttt{Z1*Z2} for \texttt{Z1=1} \cite{Hob05a}\\
             \> 100*\texttt{Z1*Z2} for \texttt{Z1=2} \\
	     \> 5 otherwise \\
    Range:   \> arbitrary
  \end{keytab}
\end{keydescription}

\begin{keydescription}{\texttt{EMINSE} --- Minimum energy for electronic
    stopping}
%
  This parameter specifies the energy below which electronic stopping
  vanishes. For energies below \texttt{EMINSE} the electronic stopping
  power $S_\mathrm{e}$ is set to zero. For energies above \texttt{EMINSE},
  \texttt{EMINSE} is subtracted from the particle energy for the evaluation
  of the stopping power. The units of \texttt{EMINSE} are eV.
  \begin{keytab}
    Type:    \> index variable array (\texttt{ATOM2},\texttt{ATOM1}) 
                of real \\
    Default: \> 0 \\
    Range:   \> any
  \end{keytab}
\end{keydescription}

\begin{keydescription}{\texttt{ENL} --- Energies for nonlocal fractions}
%
  This parameter specifies the energies $E^\mathrm{nl}$ to which the
  nonlocal fraction \texttt{XNL} values correspond. The units of \texttt{ENL}
  are eV. \texttt{ENL=0} means the energy at the stopping power maximum. It 
  should not be used with \texttt{XNLMODEL='E'}. If it is, it is allowed only as 
  the last \texttt{ENL} value, and $x^\mathrm{nl}$ at the stopping power
  maximum will not equal the last \texttt{XNL} value. Instead, $x^\mathrm{nl}$
  at the last but one \texttt{ENL} value will behave similarly as for 
  \texttt{XNLMODEL='Se'}. If \texttt{ENL} values for the same atom indices 
  appear on different records, later specifications override earlier ones.
  \begin{keytab}
    Type:    \> index variable array (\texttt{ATOM2},\texttt{ATOM1}) 
                of arbitrary-size simple arrays of real \\
    Default: \> see \texttt{XNL} \\
    Range:   \> $\ge 0$, increasing
  \end{keytab}
\end{keydescription}

\begin{keydescription}{\texttt{ESTRAG} --- Kink energy for electronic
    straggling}
%
  This parameter specifies the energy where the low- and high-energy
  asymptotic behaviors of the electronic straggling intersect. At low
  energies the straggling is proportional to the energy, while at high
  energies it is constant and equal to the Bohr straggling.
  \begin{keytab}
    Type:    \> index variable array (\texttt{ATOM2},\texttt{ATOM1}) 
                of real \\
    Default: \> 100~keV $\times$ mass of \texttt{ATOM1} \\
    Range:   \> $\ge 0$
  \end{keytab}
\end{keydescription}

\begin{keydescription}{\texttt{FACSCR} --- Factor for the screening length}
%
  This parameter specifies the factor $f$ to be applied to the
  Oen-Robinson screening length if \texttt{FAC2SCR} is not specified.
  If \texttt{FAC2SCR} is specified, it specifies $f$ in the low-energy
  limit.
  \begin{keytab}
    Type:    \> index variable array (\texttt{ATOM2},\texttt{ATOM1}) 
                of real \\
    Default: \> $1.37 / \texttt{Z1}^{1/3}$ \\
    Range:   \> $> 0$
\end{keytab}
\end{keydescription}

\begin{keydescription}{\texttt{FAC2SCR} --- Factor for the screening length
  at $x^\mathrm{nl}=1$}
%
  This parameter specifies a linear dependence of the factor $f$ for the 
  screening length on the nonlocal fraction $x^\mathrm{nl}$. If 
  \texttt{FAC2SCR} is specified, \texttt{FACSCR} is considered the value 
  of $f$ at $x^\mathrm{nl} = 0$ and \texttt{FAC2SCR} the value 
  of $f$ at $x^\mathrm{nl} = 1$. 
  \begin{keytab}
    Type:    \> index variable array (\texttt{ATOM2},\texttt{ATOM1}) 
                of real \\
    Default: \> \texttt{FACSCR} \\
    Range:   \> $\ge 0$
  \end{keytab}
\end{keydescription}

\begin{keydescription}{\texttt{LAPSIS} --- Apsis flag}
%
  This parameter specifies whether the apsis (distance of closest
  approach) of the collision (\texttt{LAPSIS=T}) or the impact
  parameter (\texttt{LAPSIS=F}) is used in the local electronic
  stopping model. Using the apsis is physically more sound, while using the
  impact parameter may save some computation time.
  \begin{keytab}
    Type:    \> logical \\
    Default: \> \texttt{T} if \texttt{LRCOIL=T} and \texttt{ELIMKP$<$100} \\
             \> \texttt{F} otherwise \\
    Range:   \> \texttt{T}, \texttt{F}
  \end{keytab}
\end{keydescription}

\begin{keydescription}{\texttt{LOCMODEL} --- Local electronic stopping model}
%
  This parameter specifies the model for the local electronic stopping power.
  It can be either \texttt{LOCMODEL='OR'} for a modified Oen-Robinson model,
  \texttt{LOCMODEL='Firsov'} for a modified Firsov model, or
  \texttt{LOCMODEL='none'} for pure nonlocal electronic stopping.
  \begin{keytab}
    Type:    \> index variable array (\texttt{ATOM2},\texttt{ATOM1}) 
                of character strings \\
    Default: \> \texttt{'OR'} \\
    Range:   \> \texttt{'OR'}, \texttt{'Firsov'}, \texttt{'none'}.
  \end{keytab}
\end{keydescription}

\begin{keydescription}{\texttt{LTAB} --- Stopping table flag}
%
  This parameter specifies whether tables of the nonlocal and local electronic
  stopping powers shall be written to the file \texttt{SE} for each ion 
  species. The energies are equally spaced on a logarithmic
  scale between \texttt{EF} of the \texttt{\&SNPAR} record and
  \texttt{ENERGY} of the \texttt{\&IONS} record.  Units are eV for the
  energy and eV/\AA\ for the stopping power.  
  \begin{keytab}
    Type:    \> logical \\
    Default: \> \texttt{F} \\
    Range:   \> \texttt{T}, \texttt{F}
  \end{keytab}
\end{keydescription}

\begin{keydescription}{\texttt{POWFIRS} --- Power in Firsov model}
%
  This parameter specifies the power used in the Firsov model that
  describes the impact parameter dependence of the electronic stopping.
  \texttt{POWFIRS} is only used with \texttt{LOCMODEL='Firsov'}.
  \begin{keytab}
    Type:    \> index variable array (\texttt{ATOM2},\texttt{ATOM1}) 
                of real \\
    Default: \> 5 \\
    Range:   \> $> 2$
  \end{keytab}
\end{keydescription}

\begin{keydescription}{\texttt{POWINT} --- Power for Interpolation of 
    Stopping Powers}
%
  This parameter specifies the constant used to interpolate the
  modified Lindhard Bethe-Bloch stopping powers.  
  \texttt{POWINT} is only effective if \texttt{SEMODEL='Lindhard'}.
  \begin{keytab}
    Type:    \> index variable array (\texttt{ATOM2},\texttt{ATOM1}) 
                of real \\
    Default: \> according to \cite{I0104} for ions specified there \\
             \> 1 otherwise \\
    Range:   \> $> 0$
  \end{keytab}
\end{keydescription}

\begin{keydescription}{\texttt{POWLIN} --- Power for low energy
    electronic stopping}
%
  This parameter specifies the power $p$ in the energy dependence of
  the Lindhard stopping power.  \texttt{POWLIN} is only effective for
  \texttt{SEMODEL='Lindhard'}. Note that changing \texttt{POWLIN} 
  will leave the electronic stopping power at 1~eV constant. Since this 
  is probably not what you want, you have to change \texttt{CORLIN} 
  accordingly. 
  \begin{keytab}
    Type:    \> index variable array (\texttt{ATOM1}) of real \\
    Default: \> 0.5 \\
    Range:   \> $0 < \texttt{POWLIN} \le 1$
  \end{keytab}
\end{keydescription}

\begin{keydescription}{\texttt{PTAB} --- Impact parameter for stopping
    power tabulation}
%
  This parameter specifies the impact parameter to be used for the local
  part of the electronic energy loss. It is only used for writing the
  third column of the stopping power table on the file \texttt{SE} if the 
  flag \texttt{LTAB} is set.
  \begin{keytab}
    Type:    \> real \\
    Default: \> 0 \\
    Range:   \> $\ge 0$
  \end{keytab}
\end{keydescription}

\begin{keydescription}{\texttt{SEFILE} --- Random electronic stopping file}
%
  This parameter specifies the name of the random electronic stopping
  file. \texttt{SEFILE} is only effective if \texttt{SEMODEL='table'}.
  \begin{keytab}
    Type:    \> index variable array (\texttt{ATOM2},\texttt{ATOM1}) 
                of character strings \\
    Default: \> ~ -- ~~~ (obligatory if \texttt{SEMODEL='table'}) \\
    Range:   \> any name ($\le$ 80 characters) of a file containing random
    electronic stopping \\
             \>data.
  \end{keytab}
\end{keydescription}

\begin{keydescription}{\texttt{SEMODEL} --- Electronic stopping model}
%
  This parameter specifies the electronic stopping power model.  
  \texttt{SEMODEL='Lindhard'} selects the combined modified Lindhard and
  Bethe-Bloch model. 
  \texttt{SEMODEL='SRIM'} selects the SRIM-2013 model.
  For \texttt{SEMODEL='table'} the electronic stopping data 
  ($-\mathrm{d}E/\mathrm{d}x$ [eV/\AA]) will be read from the file 
  \texttt{SEFILE}.
  The options \texttt{SEMODEL='ZBL'} for the ZBL85 stopping power and 
  \texttt{SEMODEL='KKK'} for the Konac-Klatt-Kalbitzer model are generally
  not recommended.
  \texttt{SEMODEL='KKK'} is only allowed for \texttt{ATOM2} corresponding to 
  Si atoms.
  \texttt{SEMODEL='none'} can be used to switch off electronic stopping.  
  \begin{keytab}
    Type:    \> index variable array (\texttt{ATOM2},\texttt{ATOM1}) 
                of character strings \\
    Default: \> \texttt{'Lindhard'} if there is a default value for 
                \texttt{CORLIN} \\
             \> \texttt{'SRIM'} otherwise \\
    Range:   \> \texttt{'Lindhard'}, \texttt{'SRIM'}, \texttt{'table'},
                \texttt{'ZBL'}, \texttt{'KKK'}, \texttt{'none'}
  \end{keytab}
\end{keydescription}

\begin{keydescription}{\texttt{STRAGGLE} --- Electronic straggling flag}
%
  This parameter specifies whether electronic straggling according to
  Konac, Klatt, and Kalbitzer (KKK) shall be taken into account. The
  KKK model may be modified using the \texttt{ESTRAG} parameter.
  \begin{keytab}
    Type:    \> index variable array (\texttt{ATOM2}, \texttt{ATOM1})
                of character strings \\
    Default: \> \texttt{'on'} if \texttt{ESTRAG} is specified \\
             \> \texttt{'off'} otherwise \\
    Range:   \> \texttt{'on'}, \texttt{'off'}
  \end{keytab}
\end{keydescription}

\begin{keydescription}{\texttt{XNL} --- Nonlocal fraction}
%
  This parameter specifies the nonlocal fraction $x^{nl}$. If \texttt{ENL}
  values are specified, the same number of \texttt{XNL} values have to be 
  specified, which then correspond to the respective \texttt{ENL} values. 
  If \texttt{ENL} values are not specified, not more \texttt{XNL}
  values must be specified than default values of \texttt{ENL} exist (usually
  2). If only one \texttt{XNL} value is given, the nonlocal fraction of
  electronic stopping is independent of energy. If \texttt{XNL} values for
  the same atom indices appear on different records, later specifications
  override earlier ones.
  \begin{keytab}
    Type:    \> index variable array (\texttt{ATOM2},\texttt{ATOM1}) 
                of arbitrary-size simple arrays of real \\
    Default: \> 0.15, 0.32 at 1eV, 1keV for \texttt{Z1=5}, \\
             \> 0.25, 0.50 at 1eV, 1MeV for \texttt{Z1=15}, \\
             \> according to \cite{I0104} for ions specified there, \\
             \> 0.5 otherwise \\
    Range:   \> $\ge 0$
  \end{keytab}
\end{keydescription}

\begin{keydescription}{\texttt{XNLMODEL} --- Nonlocal fraction model}
%
  This parameter specifies the model for the energy dependence of the
  nonlocal fraction $x^\mathrm{nl}$ (see parameter \texttt{XNL}). For
  \texttt{XNLMODEL='E'}, the nonlocal fraction is proportional to a power
  of the projectile energy $E$ (Eq.~\ref{eq6}), limited to 
  $x^\mathrm{nl} \le 1$. For \texttt{XNLMODEL='Se'}, the nonlocal fraction 
  is proportional to a power of the electronic stopping power (Eq.~\ref{eq6a}).
  \texttt{XNLMODEL='E'} is slightly more efficient than \texttt{XNLMODEL='Se'}
  \begin{keytab}
    Type:    \> index variable array (\texttt{ATOM2},\texttt{ATOM1}) 
                of integer \\
    Default: \> \texttt{'E'} for $\texttt{E0} < 10~$keV$\cdot$(ion atom mass), \\
             \> \texttt{'Se'} otherwise \\
    Range:   \> \texttt{'E'}, \texttt{'Se'}
  \end{keytab}
\end{keydescription}

n