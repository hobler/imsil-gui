In this chapter, the data files which are required or produced by IMSIL are
listed, and their format is specified. If IMSIL is run by calling its executable
directly, all files are searched for or generated in the current working
directory (i.e. from where the executable is called). Normally, IMSIL requires
files from the tables directory, so the required tables (or subdirectories
of the tables directory) must be copied or linked to the current working
directory.

It is recommended to use the Python scripts \texttt{imsil.py} and
\texttt{mimsil.py} as described in Section~\ref{s:scripts_run} to start IMSIL.
If these scripts are used and the name of the input file is \texttt{<xxx>.inp},
a directory \texttt{<xxx>} is created and \texttt{<xxx>.inp} is copied to
\texttt{<xxx>/INP}. The same is done with all other files \texttt{<xxx>.*}. In
addition, the script changes to the new directory \texttt{<xxx>}, so it
becomes the working directory. In addition, the paths to the tables directory
and the input file directory (the directory of \texttt{xxx.inp}) are passed to
IMSIL. When IMSIL attempts to read a file, it first searches the working
directory (\texttt{<xxx>}), then the input file directory,and finally the
tables directory. As an exception, the \texttt{INP} file is only searched for
in the work directory(it should always be found). The path to the tables
directory is hard-coded in the \texttt{imsil.py} script. Output files are
always generated in the working directory.

After completion of an IMSIL run, all files \texttt{<EXT>} in the working
directory \texttt{<xxx>} (where \texttt{<EXT>} stands for an arbitrary file
name) are copied to the input file directory and renamed to
\texttt{<xxx>.<ext>}. E.g., if \texttt{example1.inp} is an input file and the
imsil script is run with \texttt{python <path-to-imsil.py> example1.inp}, then
a subdirectory \texttt{example1} is created and the file \texttt{example1.inp}
is copied to \texttt{example1/INP}, the current working directory is changed to
\texttt{example1}, and IMSIL is run. The simulation generates an output file
\texttt{OUT} in the working directory, which is then copied to the input file
directory as \texttt{example1.out}. The same is done with any other files that
are created during the run.

If execution of IMSIL is delegated to a remote computer, the script copies files
obeying the pattern \texttt{<xxx>.*} in the input file directory (where the name
of the input file is \texttt{<xxx>.inp}) as well as the tables directory with
its files to the remote computer. All other files have to be copied by hand
before running IMSIL.
