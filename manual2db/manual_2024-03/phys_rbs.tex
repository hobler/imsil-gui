The code uses the principle of close encounter probability \cite{I7102} and
approximates the paths of the backscattered particles by straight trajectories
in a random medium. The program gives as the output the RBS signal as a function
of depth or backscattering energy. The nuclear encounter probability is the
cross section for the process times the integral of the product of the ion flux
and the probability of a target atom being at the point of passage.

The Rutherford backscattering cross section relates collision densities to
target atom densities. The probability of the collision is proportional to the
cross section area of the target sphere perpendicular to the path of the
projectile. The resulting probability that an ion will be scattered into a
detector with a given solid angle is represented in derivative form, the
derivative of the cross-section with respect to a differential solid angle
evaluated at a particular solid angle and projectile energy. The resulting
Rutherford differential cross section is given by:
%
 \begin{equation}
   \frac{\partial \sigma}{\partial \Omega}= 
   \left[\frac{Z_1 Z_2 e^{2}}{2 E} \right]^{2} 
   \cdot \frac {1}{\sin^{4}\psi} \cdot
   \frac{\left[\sqrt{1-{(\frac{m_1\cdot \sin\psi}{m_2}})^{2}} + \cos\psi \right]^{2}}
   {\sqrt{1-{(\frac{m_1\cdot \sin\psi}{m_2}})^{2}}}
   \label{rbs:eq1}
 \end{equation}
%
$e^2=1.44 \times 10^{-13}$~MeV$\cdot$cm is constant, $E$ and $m_1$ are the
energy and the mass of the incident ion, respectively, $m_2$ is the mass of the
target atom, and $\psi$ is the labaratory backscattering angle.  As the result
of a backscattering event, a particle will loose energy. The ratio of the
projectile energy immediately after scattering to the energy immediately before
scattering is the kinematic factor:
%
\begin{equation}
  K = \left[\frac{\sqrt{1-(\frac{m_1\cdot \sin\psi}{m_2})^{2}} 
      + \frac{m_1 \cdot \cos\psi }{m_2}}
    {1 + \frac{m_1}{m_2}} \right]^{2}
  \label{rbs:eq2}
\end{equation}
%
Atoms are assumed to vibrate independently, where the displacement probability 
is Gaussian distributed,
%
\begin{equation}
  P = \frac{1}{2 \pi x_\mathrm{rms}^{2}} \cdot \exp(-\frac{r^{2}}{2
  x_\mathrm{rms}^{2}}),
  \label{rbs:eq3}
\end{equation}
%
where $x_\mathrm{rms}$ denotes the RMS vibration amplitude of the target atoms
in one direction, and $r$ the distance vector between collision point and target
atom.  The energy detected in the detector (the backscattered energy) is
calculated by approximating the path of the backscattered particle by a
straight line in random medium.  The backscattering energy is scored in an
energy spectrum:
%
\begin{equation}
  P_\mathrm{RBS} = \frac{\partial \sigma}{\partial \Omega} \cdot P \cdot
  N_\mathrm{d}
  \label{rbs:eq4}
\end{equation}
%
where $N_\mathrm{d}$ is the ion dose. The units of \texttt{$P_\mathrm{RBS}$} are
cm$^{-2}$sr$^{-1}$.
%
\begin{center}
\begin{tabular}{lll}
   parameter \quad & IMSIL name & to be specified in record \\
   \hline
   backscattering angle $\psi$  & {\tt TILTA}   & {\tt \&OUTPUT}   \\
   dose $N_\mathrm{d}$          & {\tt DOSE}    & {\tt \&IONS}     \\
   thermal vibration amplitude $x_\mathrm{rms}$ & {\tt VIB}    
                                                & {\tt \&MATERIAL} \\
\end{tabular}
\end{center}
