The number of atoms displaced from their lattice sites may be calculated
in IMSIL either using the modified Kinchin-Pease model \cite{I7504} or by
following the recoils. In the latter case, the recoils may be followed 
until they come to rest, or until their energy falls below a user-defined
value $E_\mathrm{lim}$. When the recoil energy falls below $E_\mathrm{lim}$, the
number of interstitials $\nu_\mathrm{gen}$ generated in the remaining part of its 
collision cascade is calculated by the modified Kinchin-Pease model.

In a recoil cascade, when the energy transfered to a target atom (recoil)
exceeds its displacement energy $E_\mathrm{d}$, it is set into motion, and a
vacancy is generated (which may be subject to recombination). When the recoil
comes to rest, an interstitial is generated (which may be subject to
recombination as well), unless it produces a recoil of the same type in its last
collision (this is called a ``replacement collision'').  

Recombination with point defects generated within the same recoil 
cascade is treated by the recombination factor $f_\mathrm{rec}$.
Recombination with point defects generated by previous recoil cascades is
taken into account with a probability proportional to the point defect 
concentration. When the recoil motion is followed until the end of the 
trajectory, the damage increments are calculated in the following way
\cite{hobler_net_1996}: 
%
%
\begin{enumerate}
\item[(i)]
When an interstitial is recoiled, the local interstitial number decreases
by one. No other processes are assumed to be induced.
%
\begin{equation}
   \Delta n_\mathrm{I} = - 1 \qquad \Delta n_\mathrm{V} = 0
\end{equation}
%
\item[(ii)]
When a lattice atom is recoiled, the model takes into account defect 
recombination both with interstitials generated within the same 
cascade (described by $f_\mathrm{rec}$) and with interstitials originating from 
previous cascades (described by the probability $1-kN_\mathrm{I}$ that the 
vacancy is not located within the capture radius of an interstitial). 
%
\begin{equation}
   \Delta n_\mathrm{V} = f_\mathrm{rec} (1-kN_\mathrm{I}) \qquad 
   \Delta n_\mathrm{I} = \Delta n_\mathrm{V} - 1
   \label{eq:rgl}
\end{equation}
%
The second relation is implied by the conservation of particle numbers
during recombination.
%
\item[(iii)]
When a recoil comes to rest, it is only allowed to recombine with vacancies 
from previous cascades (described by the factor $1-kN_\mathrm{V}$), 
but not with those of the same cascade. The latter 
would mean to count recombination processes within the recoil cascade twice
and has to be excluded therefore.
%
\begin{equation}
   \Delta n_\mathrm{I} = 1-kN_\mathrm{V} \qquad 
   \Delta n_\mathrm{V} = \Delta n_\mathrm{I} - 1
   \label{eq:rs}   
\end{equation}
%
\end{enumerate}
%
$k$ is set equal to $1/N_\mathrm{sat}$ if recombination with defects from 
previous cascades is assumed to take place, where $N_\mathrm{sat}$ denotes the
saturation density of the point defects. $k=0$ is used otherwise. Notice that
$n$ denotes defect numbers, while $N$ denotes defect concentrations.

The above formulae assume that the additional Frenkel pairs described by
$f_\mathrm{rec}$ are generated where a lattice atom is recoiled. It
is also possible that part or all of the additional Frenkel pairs are
generated where the recoil stops. Introducing the fraction of these
additional Frenkel pairs generated at the starting point of the recoil
trajectory as the parameter $f_\mathrm{rg}$, Eqs.~(\ref{eq:rgl}) and 
(\ref{eq:rs}) are replaced by
%
\begin{equation}
   \Delta n_\mathrm{V} = (1 - f_\mathrm{rg} + f_\mathrm{rg} f_\mathrm{rec}) 
   (1-kN_\mathrm{I}) \qquad 
   \Delta n_\mathrm{I} = \Delta n_\mathrm{V} - 1
   \label{eq:rgl2}
\end{equation}
%
and
%
\begin{equation}
   \Delta n_\mathrm{I} = (f_\mathrm{rg} + (1-f_\mathrm{rg}) f_\mathrm{rec})
   (1-kN_\mathrm{V}) \qquad 
   \Delta n_\mathrm{V} = \Delta n_\mathrm{I} - 1
   \label{eq:rs2}   
\end{equation}
%
respectively. Note that these equations reduce to Eqs.~(\ref{eq:rgl}) and 
(\ref{eq:rs}) for $f_\mathrm{rg}=1$.


When the modified Kinchin Pease model is used to calculate the number of point
defects generated in the remaining subcascade, a formula based on the above
assumptions is used taking the probability of the three events into account.%
\footnote{The formula is similar to  
%
\begin{displaymath}
   \nu_\mathrm{st} = \nu_\mathrm{gen} \cdot f_\mathrm{rec} \cdot
      (1 - N_\mathrm{V} / N_\mathrm{sat} )
\end{displaymath}
%
which was used in earlier versions of IMSIL. $\nu_\mathrm{st} (= 
\Delta n_\mathrm{I} = \Delta n_\mathrm{V})$ denotes the average number of 
stable point defects.}

Values of $f_\mathrm{rec}$ larger than unity indicate that there is in
fact no damage recombination but rather damage multiplication due to
nonlinear effects. Damage recombination and damage saturation are
usually observed only for light ions. IMSIL also offers the possibility to turn
the target into the amorphous state when the damage density exceeds a user
specified value $N_\mathrm{amo}$.

Values of $f_\mathrm{rec}$, $N_\mathrm{sat}$, $N_\mathrm{amo}$ have been
determined for B \cite{hobler_boron_1995}, P \cite{hobler_net_1996,
herzog_monte_1995}, As \cite{hobler_net_1996, simionescu_modeling_1995}, and
BF$_2$ \cite{hobler_net_1996, simionescu_monte_1995} in Si. Further
investigations of the damage model are reported in
\cite{hobler_verification_1996, simionescu_comparison_1995}. Regarding B in Si, 
investigation of ultra-low energy B implants led to the conclusion that a
higher value of $f_\mathrm{rec}$ and no damage saturation is preferable at low energies
\cite{hobler_modeling_1997}. These values are also  consistent with the dose
dependence of higher-energy implantations in (100)-Si. The default values in
IMSIL have been adjusted accordingly. It should be noted, however, that one
should revert to the values given in \cite{I9504} if the dose dependence of
[110] channeling is investigated. 

The model may also be used in a simplified version that does not consider
defect recombination with previous cascades, corresponding to $k=0$ and/or
$N_\mathrm{I}=N_\mathrm{V}=0$. It is also possible that the damage 
distributions are only updated after a certain number of cascades. 

As an alternative in case of true recombination ($f_\mathrm{rec} < 1$), the
capture radius model may be used \cite{simionescu_comparison_1995}. In this
model the positions of vacancies and interstitials generated in each collision
cascade are stored, and vacancies and interstitials that are within a capture
radius $r_\mathrm{cap}$ are annihilated. 

As an alternative in case of damage multiplication ($f_\mathrm{rec} > 1$), the
amorphous pocket model may be used \cite{hobler_amorphous_2003}: In this
model, each energy deposition event is assigned a sphere the volume of which
would be molten according to the heat of fusion (0.52 eV/atom in Si), if surface
effects could be excluded. After each collision cascade, the connected volume
of overlapping melting spheres are determined. Each such overlapping area defines
an amorphous pocket candidate. The number of displaced atoms associated with it 
is the ratio of the sum of energies deposited by their members and the heat
of fusion. Surface effects are taken into account by shrinking the amorphous 
pocket candidate by a certain amount. If the remaining pocket meets a minimum 
size requirement, an amorphous pocket is generated with the number of displaced 
atoms calculated as indicated above. Otherwise the amorphous pocket candidate is
abandoned, and the interstitials contained in it are used to calculate the
number of displaced atoms. 

In addition to the above model, an ion beam induced interfacial amorphization
(IBIIA) model \cite{I9766} can be activated. In this model, existing amorphous
zones grow with a rate proportional to nuclear energy deposition.

In addition to being recorded, damage to crystalline regions can also be
considered in the selection of collision partners.

\begin{center}
\begin{tabular}{lll}
   parameter \quad  & IMSIL name & to be specified in record \\
   \hline
   flag for considering defect recombination  &                               \\
   with previous cascades and using damage in & {\tt LDAMDYN}& {\tt \&SETUP}  \\
   crystalline regions                        &                               \\
   damage update interval                     & {\tt NIONHIS}& {\tt \&SETUP} \\
   refined damage model flag                  & {\tt LDAM}   & {\tt \&DAMAGE} \\
   displacement energy $E_\mathrm{d}$         & {\tt ED}     & {\tt \&DAMAGE} \\
   recoil cutoff energy $E_\mathrm{lim}$      & {\tt ELIMKP} & {\tt \&DAMAGE} \\
   replacement collision flag                 & {\tt LREPL}  & {\tt \&DAMAGE} \\
   recombination fraction $f_\mathrm{rg}$     & {\tt FRACRG} & {\tt \&DAMAGE} \\
   damage recombination factor $f_\mathrm{rec}$ & {\tt FREC} & {\tt \&DAMAGE} \\
   damage saturation density $N_\mathrm{sat}$ & {\tt DAMSAT} & {\tt \&DAMAGE} \\
   amorphization threshold $N_\mathrm{amo}$   & {\tt DAMAMO} & {\tt \&DAMAGE} \\
   capture radius model flag                  & {\tt LCAP}   & {\tt \&DAMAGE} \\
   capture radius                             & {\tt RCAP}   & {\tt \&DAMAGE} \\
   amorphous pocket model flag                & {\tt LPOCK}  & {\tt \&DAMAGE} \\
   heat of fusion                             & {\tt EMELT}  & {\tt \&DAMAGE} \\
   amorphous pocket shrink parameter          & {\tt FPOCK}  & {\tt \&DAMAGE} \\
   minimum pocket size                      & {\tt NPOCKMIN} & {\tt \&DAMAGE} \\
   IBIIA model flag                           & {\tt LIBIIA} & {\tt \&DAMAGE} \\
   IBIIA growth rate constant                 & {\tt RIBIIA} & {\tt \&DAMAGE} \\
\end{tabular}
\end{center}
