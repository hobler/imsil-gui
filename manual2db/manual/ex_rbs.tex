This example shows the input file for the simulation of an RBS/C spectrum
of a 1 $\mu$m thick (100) crystalline silicon membrane with a damage profile
defined in a histogram input file. 3~MeV He ions aligned with the [001]
direction are used.

\begin{verbatim}
RBS/C spectrum calculation  
He in <100> Si, 3000 keV, 0 Tilt, 3D interstitial
 &setup    nr=1 natom=2 
           atom1=2 atom2=2 ndim=1 usehis=t hisfile='dam2percent.his' /  
 &ions     name='He' energy=3000 dose=1e14 tilt=0.0 nion=10000 /
 &material region=1 name='Si' xtal='on' /
 &damage   ldam=t lamo2=f /
 &geometry posif=0,10000 /
 &output   lhis=t lrbs=t wboxe=6000. nboxe=0 tilta=10. / 
\end{verbatim}

The histogram file is read on the \texttt{\&setup} record (\texttt{usehis=t
hisfile='dam2percent.his'}). The \texttt{.his} file must conform to the format
of a 1-D histogram file (see Section~\ref{s:his1d}), and can of course be an
output histogram of another IMSIL run. In this case it has been created by hand
with a constant damage concentration of 2\% (not shown). Note that
\texttt{hisfile} must refer to the parent directory if IMSIL is run with the
\texttt{imsil.py} script, since this file is not copied into the subdirectory
where IMSIL is run. \texttt{atom1=2} indicates that the second atom stored in
the histogram file is to be read. \texttt{atom2=2} indicates that the damage
concentrations of atom 2 of the simulation are to be initialized with these
values.

\texttt{lamo2=f} on the \texttt{\&damage} record specifies that the 3-D random
interstitial model be used. In this model, the target atoms are displaced due to
lattice vibration according to a 3-D Gaussian probability density function. The
default is displacement in the plane perpendicular to the direction of motion
according to a 2-D Gaussian probability density function (\texttt{lamo2=f}),
which is computationally slightly more efficient.

Finally, \texttt{lrbs=t} on the \texttt{\&output} record causes an RBS file to
be written, see Section~\ref{s:his1d}. The energy box widths are 6000\AA. The
backscattering angle is specified as 170\textdegree\ (\texttt{tilta=10.}).
