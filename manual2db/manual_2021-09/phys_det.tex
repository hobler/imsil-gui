Note: This model has not been used for a long time. It might not be fully
functional.

Standard models based on the picture that damage consists of displaced atoms
surrounded by the ideal lattice neglect the relaxation of the lattice around
the defects and therefore can lead to the overestimation of implant damage
when analyzed by RBS/C\nocite{I6902}.  In addition, because of the assumption
that the atoms are randomly displaced within the lattice these models often fail
to describe multiaxial RBS/C measurements of implaned Si \cite{Lul04}. 
Atomic-scale models \cite{Ric04,Leu99,Koh99} have significantly improved the
understanding of structure and properties of small native defects in silicon. 
Such calculations yield the atomic positions at strictly defined positions,
corresponding to energy minima, rather than at random positions.  In addition,
these defects cause lattice relaxation, which interacts with the analyzing beam,
increasing dechanneling and the RBS/C signal. Recent interpretation of RBS/C
measurements with atomic defect models, structurally relaxed with empirical
potentials, gave an improved interpretation of multiaxial RBS/C analysis of Si
containing low levels of disorder \cite{Lul04,Bal02}.  As an alternative to the
statistical damage model a deterministic model of damage which considers the
exact atom positions of each defect and the lattice strain around them has been
implemented in IMSIL.  

Using a Si cell with a size of 10x10x10 unit cells relaxed with the Tersoff
III \cite{I8848} empirical potential we have calculated the coordinates of
small interstitial clusters $I_{n}$, where {n} denotes the number of excess
atoms ($n=1 \ldots 4$).  The relaxation was done by a molecular dynamics
simulation of quenching to 0~K in 5~ps.  These defects were subsequently
introduced into a 216 atoms Si cell and relaxed with the ab-initio code VASP
\cite{Kre96}. In both cases we determined the positions of the defect atoms as
well as the strain associated with the defects. Only displacements exceeding
0.0036~\AA\ were considered. 

A big crystalline simulation cell with user defined column width is
populated with clusters which contain not only the defects, but also the
strained regions around them.  The desired defect concentration is defined
by specifying the desired atom index of a histogram or backup file. Also, small
defect clusters can be placed at user defined positions using a defect file. If 
a 1-D concentration profile is used, the defect type can be defined by
specifying the fraction of various defect types relative to all defects. 
For the population of the cell from histograms, presently only one defect type
at a time can be specified. Defect coordinates can be chosen from either
ab-initio or classical molecular dynamics (\texttt{LSTRMD=T} calculations.  The
defects are inserted as to guarantee a minimum distance between them but
otherwise at random positions, considering all (24) symmetry-equivalent
orientations for the silicon crystal lattice. An equivalent number of vacancies
is also inserted into the cell in order to balance the interstitial profile and
to keep the number of atoms constant. The strain of nearby defects is superposed
linearly.  

When recoils come to rest, they will be assigned to the nearest lattice site
and, depending on the number of atoms at this site and on the user defined
defect type fractions, the appropriate interstitial configuration will be
formed.  Possible cluster formation reactions are
%
\begin{equation}
    \mathrm{V} + \mathrm{V}_{n} \rightarrow \mathrm{V}_{n+1}, \hspace{1cm} 
    \mathrm{I} + \mathrm{I}_{n} \rightarrow \mathrm{I}_{n+1}, 
\end{equation}
%
and cluster dissolution reactions
%
\begin{equation}
    \mathrm{V} + \mathrm{I}_n \rightarrow \mathrm{I}_{n-1},
    \hspace{1cm} \mathrm{I} + \mathrm{V}_{n} \rightarrow
    \mathrm{I}_{n-1} ,
\end{equation}
%
where V and I denote vacancies and interstitials, respectively, and 
$n$ is the number of vacancies or interstitial atoms in the cluster.

In order to keep simulation times reasonable, defects from ``old'' cascades are
restricted to a columnar domain with lateral dimensions which correspond to
column width. The starting points of the ion trajectories are
randomly generated in the intersection of the surface and the column, but the
collision cascade is allowed to develop in the whole target. Defects around
the ion and recoil trajectories are generated by assuming periodicity in the
lateral directions. The size of the column in lateral directions determines
the statistical quality of the results and the computation time.

At the end of the simulation, deterministic defects can be printed into a
defects file if desired. Presently only 1-D layered structures are allowed with
deterministic damage model simulations. 

\begin{center}
\begin{tabular}{lll}
   parameter \quad & IMSIL name & to be specified in record \\
   \hline
   deterministic defect mode flag    & {\tt LDET}     & {\tt \&DAMAGE} \\
   coulumn width                     & {\tt WCOL}     & {\tt \&DAMAGE} \\
   split-$\langle 110 \rangle$ fraction   
                                     & {\tt FRACSPL}  & {\tt \&DAMAGE} \\
   di-interstitial fraction          & {\tt FRACDI}   & {\tt \&DAMAGE} \\
   tri-interstitial fraction         & {\tt FRACTRI}  & {\tt \&DAMAGE} \\
   four-interstitial fraction        & {\tt FRACFOUR} & {\tt \&DAMAGE} \\
   tetra-interstitial fraction       & {\tt FRACTET}  & {\tt \&DAMAGE} \\
   hexagonal-interstitial fraction   & {\tt FRACHEX}  & {\tt \&DAMAGE} \\
   MD strain flag                    & {\tt LSTRMD}   & {\tt \&DAMAGE} \\
   use backup flag                   & {\tt USEBK}    & {\tt \&SETUP}  \\
   backukp file                      & {\tt BKFILE}   & {\tt \&SETUP}  \\
   use histogram flag                & {\tt USEHIS}   & {\tt \&SETUP}  \\
   histogram file                    & {\tt HISFILE}  & {\tt \&SETUP}  \\
   use defect file flag              & {\tt USEDEF}   & {\tt \&SETUP}  \\
   deterministic file                & {\tt DEFFILE}  & {\tt \&SETUP}  \\
   atom index on histogram file      & {\tt ATOM1}    & {\tt \&SETUP}  \\
   atom index in simulation          & {\tt ATOM2}    & {\tt \&SETUP}  \\
   deterministic defects output flag & {\tt LDEF}     & {\tt \&OUTPUT} \\
\end{tabular}
\end{center}
