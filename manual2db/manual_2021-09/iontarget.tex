IMSIL simulates the interaction of an ion beam with a target. Throughout this
manual, \textbf{ion} means the particle that is given initial conditions by the
user. We use this term irrespective of the atom's actual charge
state.\footnote{Ions ``forget'' their initial charge state after a short
distance travelled inside the target.} The ions may be single atoms or
molecules.

\textbf{Beam} denotes the properties of a collection of ions. The term is
used here in a broader sense than in common parlance. During conventional ion
implantation or sputtering, ions are part of a ``beam''. During single-ion
implantation, the ``beam'' consists of only one ion. In this case, when many
ions are simulated, the results can be interpreted as probabilities and
probability density functions. Atoms starting from inside the target, e.g., as
recoils in nuclear reactions, are also considered a ``beam'' in the context
of this manual.

The target consists of one or several \textbf{regions} with homogeneous material
properties. A crystalline region may be modified during a simulation due to
damage formation. While this kind of simulation can be considered
\textbf{dynamic}, we reserve the word here for sputtering simulations where the
geometry and composition of the target changes as a function of time. In this
case, the target is defined by \textbf{cells} rather than regions. Cells
typically have the size of histogram bins in order to resolve the spatial and
temporal variations of the target composition as the simulation proceeds. In the
code, regions and cells are considered ``domains''.

While ``ion'' refers to a particle that receives initial conditions by
specifications in the input file, we use the terms \textbf{projectile} and
\textbf{recoil} when we talk about collisions: The projectile is the energetic
atom that impinges on another atom, the recoil, which initially is at rest and
receives energy in the collision. We use the term ``recoil'' irrespective of
whether the atom permanently leaves its original site.

