For the purpose of this manual we define the term \textbf{distribution} as the
probability density function times the yield times the implantation dose. As an
example, the nuclear energy deposition distribution $f_\mathrm{NE}(z)$ is the
probability density function of nuclear energy deposition times the total
nuclear energy deposited in the target per ion, $Y_\mathrm{NE}$, times the area
dose \texttt{DOSE} as specified on the \texttt{\&IONS} record. Since probability
density functions are normalized, 
%
\begin{equation}
    \int_{-\infty}^{\infty} f_\mathrm{NE}(z)\,\mathrm{d}z
    = Y_\mathrm{NE} \times \texttt{DOSE}
    \label{eq:distr1d}
\end{equation}
%
Internally, IMSIL's length units are {\AA}ngstr{\o}ms. Thus, an area dose
\texttt{DOSE} as specified in the \texttt{INP} file is converted to
{\AA}$^{-2}$, and the units of $f_\mathrm{NE}(z)$ are eV/{\AA}$^3$. Note that
the units of $Y_\mathrm{NE}$ are eV. eV and {\AA}ngstr{\o}ms, as well as degrees
for angular distributions, are generally used for the units of the histograms.
There is one exception to this rule, spatial atom distributions in the
\texttt{HIS} file (also known as atom concentrations) are converted to cm$^{-3}$
when written to the file. 

Eq.~(\ref{eq:distr1d}) can also be written
%
\begin{equation}
    \int_{-\infty}^{\infty} 
    \frac{ f_\mathrm{NE}(z) }{ \texttt{DOSE} } \, \mathrm{d}z
    = Y_\mathrm{NE}
    \label{eq:distr1d_a}
\end{equation}
%
When the default implantation dose of $\texttt{DOSE} =
10^{16}$~cm$^{-2} = 1$~{\AA}$^{-2}$ is used, $f_\mathrm{NE}(z) =
f_\mathrm{NE}(z) / \texttt{DOSE}$ numerically. Therefore, with the default
dose the distributions, except for the atom concentrations on the \texttt{HIS}
file, may also be considered the probability density functions times the yield.
E.g., in the case of nuclear energy deposition, $f_\mathrm{NE}(z)$ may be
interpreted as the nuclear energy deposition per ion and width $\Delta z$ of the
depth interval (units: eV/{\AA}).

The above desciption is valid if \texttt{DOSE} refers to an area dose. With
\texttt{DOSEUNITS} one may also demand \texttt{DOSE} to be interpreted as a line
dose (\texttt{DOSEUNITS='cm-1'}) or an ion number count
(\texttt{DOSEUNITS='1'}). In these cases, the area dose is calculated using the
extent of the implant area as defined by \texttt{XINIT} and \texttt{YINIT} on
the \texttt{\&IONS} record. If the extent of the implant area in one of the
directions is zero, it is replaced by unity. As a consequence, the units of
$f_\mathrm{NE}(z)$ will be different. For instance, when a line dose is
specified and \texttt{XINIT(1)=XINIT(2)}, the area dose cannot be calculated and
the line dose substituted in Eq.~(\ref{eq:distr1d}) for \texttt{DOSE} will have
units {\AA}$^{-1}$. As a result, the units of $f_\mathrm{NE}(z)$ will be 
eV/{\AA}$^2$. If the energy density per {\AA}$^3$ is to be calculated
a-posteriori, it is obtained by dividing $f_\mathrm{NE}(z)$ by the extent of the
implant area in $x$ direction. However, these values will represent only the
deposited energy or concentration inside the implant area sufficiently far from
the edges. \textbf{It is therefore generally not recommended to record 1-D
histograms when the dose is not specified as an area dose.} For completeness we
mention that in case of atoms densities, the units will be {\AA}/cm$^3$ in the
above example. 


IMSIL produces the following 1-D histogram files if the respective flag listed
in the last column of the following table is set on the \texttt{\&OUTPUT} record:
\bigskip
%
\begin{center}
\begin{tabular}{|l|p{0.65\textwidth}|l|}
\hline
Name       & File content                                           & flag \\
\hline
\tt HIS    & Spatial distribution of stopped atoms                  & \tt LHIS \\
\tt CELL   & Cell-based target description file                     & \tt LCELL \\
\tt HISEE  & Spatial distribution of deposited electronic energy    & \tt LHISEE \\
\tt HISNE  & Spatial distribution of deposited nuclear energy       & \tt LHISNE \\
\tt HISM   & Spatial distribution of displacement moments           & \tt LHISM \\
\tt HISP   & Spatial distribution of deposited momentum             & \tt LHISP \\
\tt HISD   & Displacement distribution                              & \tt LHISD \\
\tt HISIV  & I--V pair distance distribution                        & \tt LHISIV \\
\tt HISB   & Spatial distribution of backscattered atoms            & \tt LHISB \\
\tt HISEB  & Energy distribution of backscattered atoms             & \tt LHISEB \\
\tt HISAB  & Angular distribution of backscattered atoms (2-D angle)& \tt LHISAB \\
\tt HISAAB & Angular distribution of backscattered atoms (3-D angle)& \tt LHISAAB \\
\tt HIST   & Spatial distribution of transmitted atoms              & \tt LHIST \\
\tt HISET  & Energy distribution of transmitted atoms               & \tt LHISET \\
\tt HISAT  & Angular distribution of transmitted atoms (2-D angle)  & \tt LHISAT \\
\tt HISAAT & Angular distribution of transmitted atoms (3-D angle)  & \tt HISAAT \\
\tt RBS    & RBS spectra                                            & \tt LRBS \\
\hline
\end{tabular}
\end{center}

\bigskip

The histogram files are generally written at the end of the simulation. They are
written by the following FORTRAN statements:
%
\begin{verbatim}
      WRITE (22,*) 1+NSPEC,2*NBOX+2
      WRITE (22,*) NINT(E0),(NINT(Z1(I)),I=1,NSPEC)
      WRITE (22,*) Z(1),(1E-4*HISMIN,I=1,NSPEC)
      DO, IBOX=1,NBOX
         WRITE (22,*) Z(IBOX),(HIST(I,IBOX),I=1,NSPEC)
         WRITE (22,*) Z(IBOX+1),(HIST(I,IBOX),I=1,NSPEC)
      ENDDO
      WRITE (22,*) Z(NBOX+1),(1E-4*HISMIN,I=1,NSPEC)
\end{verbatim}
%
where the meaning of the variables is detailed below.

\texttt{NSPEC} denotes the number of atom species to be recorded, which depends
on the type of the histogram and \texttt{LDYN}. Let \texttt{N1} denote the
number of atom species in the ion (\texttt{N1=1} for atomic ions) and
\texttt{N2} the number of recoil species to be considered in the statistics. For
\texttt{LDYN=T}, all atom species may become recoils and must be kept track off,
so \texttt{N2=NATOM}, where \texttt{NATOM} is the number of atom species
specified on the \texttt{\&SETUP} record. For \texttt{LDYN=F}, only the atoms of
the materials may become recoils, thus \texttt{N2=NATOM-N1}. Note that for
\texttt{LDYN=T}, ion atoms are contained in both \texttt{N1} and \texttt{N2}, so
one can distinguish the properties of implanted ions from those previously
implanted. 

\texttt{NSPEC} now depends on \texttt{N1} and \texttt{N2}: In addition, for the
\texttt{HIS} file, \texttt{NSPEC} depends on two parameters of the
\texttt{\&DAMAGE} record: If \texttt{LRCOIL=T}, \texttt{NSPEC=N1+2*N2}.
Otherwise, if \texttt{LDAM=T}, \texttt{NSPEC=N1+N2}. Else \texttt{NSPEC=N1}. For
the \texttt{HISEE}, \texttt{HISNE}, and \texttt{HISD} file,
\texttt{NSPEC=N1+N2}. For the \texttt{HISM} and \texttt{HISP} file,
\texttt{NSPEC=3*(N1+N2)}. For the \texttt{HISIV} file \texttt{NSPEC=N2}, and for
the \texttt{CELL} file \texttt{NSPEC=NATOM}.

From line 2 on, the 1-D histogram files contain \texttt{NSPEC+1} columns. In
the first column, the independent variable \texttt{Z} (depth, energy, angle) is 
listed, in the other columns the histogram values for the \texttt{NSPEC} atom
species. They correspond from left to right to the ion atoms, atoms that may
become recoils (and thus interstitials), and vacancies left behind by recoils.
The atom species are ordered according to their occurrence in the ion and
material names: First the ion, and then the materials from region 1 to how many
regions are defined. Within the ion and material names from left to right. 
E.g., if the ion and 3 materials are defined by 
%
\begin{verbatim}
 &IONS NAME='B' ... /
 &MATERIAL REGION=2 NAME='SI3N4' ... /
 &MATERIAL REGION=1 NAME='SIO2' ... /
 &MATERIAL REGION=3 NAME='SI' ... /
\end{verbatim}
%
B is defined first, Si second, O third, and N fourth. Therefore, column 2 of the
\texttt{HIS} file corresponds B. If \texttt{LDAM=T} or \texttt{LRCOIL=T}, column
3 corresponds to Si, column 4 to O, and column 5 to N. If \texttt{LRCOIL=T},
column 6 corresponds to Si vacancies, column 7 to O vacancies, and column 8 to N
vacancies. To help identify the atom species, their atomic numbers are given in
line 2 of the file. Atomic numbers of vacancies are given a negative sign.%
\footnote{The columns 2 to \texttt{N1}$+$\texttt{N2}$+$1 in the histogram files
correspond to the atom indices as defined in Section~\ref{s:atom}.}

\texttt{NBOX} denotes the number of histogram boxes (bins) and correspond to
\texttt{NBOX}, \texttt{NBOXA}, \texttt{NBOXD}, or \texttt{NBOXE} of the
\texttt{\&OUTPUT} record (\texttt{NBOX*} for short). The user may choose between
two options regarding how histograms evolve during the simulation. In both
cases, the histograms are initialized with a box width of \texttt{WBOX*} of the
\texttt{\&OUTPUT} record, and are then adapted as to include all data points.
When \texttt{NBOX*} is specified as a positive number on the \texttt{\&OUTPUT}
record (or its default value is being used), the histogram is adjusted by
doubling the box width as required, while leaving the number of boxes the same.
Only nonzero histogram values are written to the histogram file at the edges, so
the number of boxes written to the file may be smaller than \texttt{NBOX}
specified on the \texttt{\&OUTPUT} record. On the other hand, if
\texttt{NBOX*=0} is specified on the \texttt{\&OUTPUT} record, the box widths
are kept constant equal to \texttt{WBOX*}, while the number of boxes is
increased as required.

From line 3 on, there are \texttt{2*NBOX+2} rows. \texttt{Z(IBOX)} and
\texttt{Z(IBOX+1)} are the lower and the upper boundary of histogram box
\texttt{IBOX}, respectively, and \texttt{HIST(I,IBOX)} denotes the distribution
function $f$. The physical meaning of \texttt{Z} and \texttt{HIST} depends on
the file. As apparent from the code, the file contains duplicate information.
This is to facilitate plotting by allowing for simply connecting the data points
to draw the histogram. The first and last data point are added to provide a
visual closure of the first and last box.

\texttt{HIS}, \texttt{CELL}, \texttt{HISEE}, \texttt{HISNE}, \texttt{HISM}, and
\texttt{HISP} contain spatial histograms with \texttt{Z} denoting the depth
coordinate. 

In the \texttt{HIS} and \texttt{CELL} file, \texttt{HIST(I,IBOX)} denotes the
concentration [cm$^{-3}$] of atom species \texttt{I} in histogram box
\texttt{IBOX}. As described above, the \texttt{HIS} and \texttt{CELL} files
differ in the atom species recorded. E.g., the \texttt{CELL} file never contains
vacancy concentrations. In addition, \texttt{HIS} and \texttt{CELL} files differ
in that the atom concentrations in the \texttt{CELL} file include all target
atoms, while in the \texttt{HIS} file they include only atoms added in the
implant step plus, optionally, those read in from a \texttt{HIS} file at the
beginning of the simulation (see the \texttt{USEHIS} and related parameters of
the \texttt{\&SETUP} record). 

In the \texttt{HISEE} and \texttt{HISNE} files, \texttt{HIST(I,IBOX)} 
contains the distributions of electronic and nuclear energy deposition,
respectively. The units are [eV/\AA$^3$]. 

The \texttt{HISM} file contains the histogram of the first redistributive moment
as a function of depth. The first redistributive moment is the sum of the
differences between end-point and initial coordinates over all recoil
trajectories that start and end inside the target \cite{hobler_crater_2018}.
Since it is a vector quantity, three columns are stored in the \texttt{HISM}
file for each atom type in adjacent columns (x, y, and z component of the
moment). The units of \texttt{HIST} are \AA. 

The \texttt{HISP} file contains the histogram of the deposited momentum as a
function of depth. Deposited momentum is the momentum of the ions and recoils
when their simulated trajectories are terminated. Since it is a vector quantity,
three columns are stored in the \texttt{HISP} file for each atom type in
adjacent columns (x, y, and z component of the momentum). The units of
\texttt{HIST} are $(m_0 \, \mathrm{eV})^{1/2}/$\AA$^3$, where $m_0$ denotes the
atomic mass unit. To obtain the momentum in SI units (kg$\,$m/s), one has to
multiply with $(m_0 \, \mathrm{eV/J})^{1/2} = 1.6311 \times 10^{-23}$.

The \texttt{HISD} file contains the displacement distribution, where \texttt{Z}
denotes the distance between origin and end point of recoil trajectories.
Similarly, the \texttt{HISIV} file contains the I-V pair distance distribution
of a cascade, where \texttt{Z} denotes the distance between interstitial and
vacancy \cite{hobler_continuum_1999}. The histogram is obtained by looping over
both the interstitials and vacancies of each cascade and collecting the
distances between them in a histogram. The I-V pair distribution function is a
probability density function. The positions of the vacancies may be restricted
to certain depth intervals (\texttt{POSIVMIN} and \texttt{POSIVMAX} of the
\texttt{OUTPUT} record). \texttt{NHISIV} such intervals may be specified.
\texttt{NHISIV} data blocks corresponding to the code snippet given above are
written to the \texttt{HISIV} file.  

The \texttt{HIS*B} and \texttt{HIS*T} files contain distributions of
backscattered and transmitted atoms, respectively. \texttt{HISB} and
\texttt{HIST} contain the spatial distributions of the ejected atoms as a
function of the lateral coordinate $x$ (i.e., in the code snippet given above,
\texttt{Z} has to be replaced by \texttt{X}). The number of species listed,
\texttt{NSPEC}, is the same as for the spatial atom distributions inside the
target, stored in the \texttt{HIS} file. For the first \texttt{N1+N2} columns,
the coordinate $x$ refers to the position where the atom leaves the surface
boundary layer, which normally is a few {\AA}ngstr{\o}ms above the surface. For
the last \texttt{N2} columns, $x$ refers to the position where the ejected atoms
originate from.  

The \texttt{HISAB}, \texttt{HISAAB}, \texttt{HISAT}, and \texttt{HISAAT} files
contain angular distributions of ejected atoms. In the \texttt{HISAB} and
\texttt{HISAT} file, the angle is between the projection of the atom's direction
of motion to the $x$-$z$ plane and the negative and positive $z$ axis,
respectively. In the \texttt{HISAAB} and \texttt{HISAAT} file, the angle is
taken between the direction of motion and the analyzer direction defined by
\texttt{TILTA} and \texttt{ROTATA} of the \texttt{\&OUTPUT} record. The units of
the angles are degrees. Depending on the \texttt{LRCOIL} parameter of the
\texttt{\&DAMAGE} record, data for \texttt{NSPEC=N1} or \texttt{NSPEC=N1+N2}
atom species are written to the files.  

The \texttt{HISEB} and \texttt{HISET} files contain the energy distributions of
the ejected atoms. The units of the energies are eV. Depending on the
\texttt{LRCOIL} parameter of the \texttt{\&DAMAGE} record, data for
\texttt{NSPEC=N1} or \texttt{NSPEC=N1+N2} atom species are written to the files.

Finally, the \texttt{RBS} file contains the energy and depth spectra of a
Rutherford backscattering simulation (\texttt{LRBS=T} on the \texttt{\&OUTPUT}
record). If the target contains only one atom species (\texttt{N2=1}), then two
blocks according to the code snippet at the beginning of this section are
written, first the energy and then the depth distribution. While in these two
blocks the contributions from all target atoms are added, thus each one
containing two columns of data (energy/depth and the distribution,
\texttt{NSPEC=1}), for \texttt{N2>1} two other blocks are written where the
contributions of different target atom species are separated
(\texttt{NSPEC=N2}). The units of energy and depth are eV and \AA, respectively.
The units of the distribution functions are eV$^{-1}$cm$^{-2}$sr$^{-1}$ and
\AA$^{-1}$cm$^{-2}$sr$^{-1}$, respectively.  
