The \texttt{read\_output.py} file may be used as a module to access the IMSIL
output data from a Python 3 script. In addition, \texttt{read\_output.py} may be
used as a script with simple plotting functionality. 

To use functions of \texttt{read\_output.py} in your own Python script, you must
import \texttt{read\_output} as a module. E.g., to read and plot data from
\texttt{example.his}, you can write
%
\begin{verbatim}
    import sys
    import matplotlib.pyplot as plt
    sys.path.insert(0, <path-to-scripts>)
    from read_output import read_his
    x, c = read_his('example.his')
    plt.plot(x, c)
    plt.show()
\end{verbatim}
%
Here, \texttt{<path-to-scripts>} denotes the path to the directory where
\texttt{read\_output.py} resides.

The following functions are defined in \texttt{read\_output.py}:
\bigskip
%
\begin{center}
\begin{tabular}{|l|p{0.65\textwidth}|}
\hline
Function           & Purpose \\
\hline
\tt read\_par      & Read the file names (without extension) defined in a
                     parameters file. Optionally, also return the value of one
                     parameter for each file name. \\
\tt read\_inp      & Read the value of a parameter from an input (\texttt{.inp})
                     file. \\
\tt read\_out      & Read a value from an output data table in the output
                     (\texttt{.out}) file. \\
\tt extract\_yield & Read the sputter yield from the output (\texttt{.out})
                     file. \\
\tt read\_his      & Read a 1-D histogram file (\texttt{.his} and other files
                     described in Section~\ref{s:his1d}). \\
\tt read\_his2     & Read a 2-D histogram file (\texttt{.his2} and other files
                     described in Section~\ref{s:his2d}). \\
\tt read\_his3     & Read a 3-D histogram file (\texttt{.his3} and other files
                     described in Section~\ref{s:his3d}). \\
\tt read\_trajectories & Read a trajectory \texttt{.tra.} file. The return value
                     is a list of cascades, where each cascade is a list of
                     trajectories, where each trajectory is a list of collision
                     points. \\
\hline
\end{tabular}
\end{center}

\bigskip
For a detailed description of the function arguments and return values see the
code.

\texttt{read\_output.py} may also be called as a script such as
%
\begin{verbatim}
    <path-to-read_output.py> example.his
\end{verbatim}
%
This has very limited functionality, though. The code in the main function of
\texttt{read\_output.py} is mainly intended to provide a template for calling
the functions in your own script.


