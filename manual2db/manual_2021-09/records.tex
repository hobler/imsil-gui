The input language of IMSIL uses the FORTRAN90 syntax for NAMELIST input.  Each
\textbf{input record} in the input file \texttt{INP} corresponds to a namelist
in the program. It contains a set of input parameters.  An input record starts
with an \texttt{\&}, followed by the name of the record (= the name of the
namelist) and the entities (parameters) to be given values. It ends with a
slash. Everything outside an input record is ignored and can be used as a
comment. Comments are typically placed at the beginning of the file. Comments
may also be placed inside a record after an exclamation mark ``!''. The comment
then extends to the end of the line.

While all names are case-insensitive, in this manual (except for the
Examples Chapter), parameter and record names are written uppercase to better
distinguish them from normal text.

A typical input record reads
%
\begin{verbatim}
 &NAME PARNAME1=value1 PARNAME2=value2 ... /
\end{verbatim}
%
\texttt{\&NAME} is the name of the record, \texttt{PARNAME<i>}
is the \texttt{i}-th parameter name, and \texttt{value<i>} the value to be assigned
to \texttt{PARNAME<i>}. Values of logical parameters may either be \texttt{T}
(true) or \texttt{F} (false). Character values have to be quoted.

Parameter specifications for a namelist may be distributed across multiple input
records:
%
\begin{verbatim}
 &NAME PARNAME1=value1 ... /
 &NAME PARNAME1=value3 PARNAME2=value2 ... /
\end{verbatim}
%
If input parameters are repeated, the later definitions override the earlier
one(s), with the exception of simple arrays of arbitrary size, see below. In the
example above, \texttt{PARNAME1=value3}.

There are two kinds of array parameters in IMSIL: First, \textbf{simple arrays}
are one-dimensional arrays of either fixed size (often the size is 2
or 3), or their size is determined by the number of elements given
(``arbitrary'' size). Simple arrays have to be specified according to the rules
for arrays in namelists, i.e.\ the elements are specified after the
``\texttt{=}'' sign separated by commas:  
%
\begin{verbatim}
 &NAME ARRAYNAME=value1,value2,value3 ... /
\end{verbatim}
%
When more than one input record is specified for the same namelist and
specifications for the same simple array of arbitrary size are given,
unless otherwise noted, later array values are appended to earlier ones. Thus,
%
\begin{verbatim}
 &NAME ARRAYNAME=value1,value2 ... /
 &NAME ARRAYNAME=value3 ... /
\end{verbatim}
%
defines the array \texttt{(value1,value2,value3)} like in the example above, if
\texttt{ARRAYNAME} is a simple array of arbitrary size. If \texttt{ARRAYNAME} is
a simple array of fixed size, then \texttt{ARRAYNAME=value3}, i.e., only the
first element of \texttt{ARRAYNAME} is given a value. The maximum number of
array values that can be specified on each input record is 20.

Second, \textbf{index variable arrays} are arrays whose values may be specified
for individual indices using index variables. For instance,
%
\begin{samepage}
\begin{verbatim}
 &MATERIAL REGION=2 NAME='SI3N4' ... /
 &MATERIAL REGION=1 NAME='SIO2' ... /
 &MATERIAL REGION=3 NAME='SI' ... /
\end{verbatim}
\end{samepage}
%
specifies the array \texttt{NAME(1:3)=('SIO2','SI3N4','SI')}, where in each
record the index variable \texttt{REGION} indicates the index of the array
element whose value is to be set.

If an index variable array appears on a record that does not contain the index
variable(s), the array values for all unspecified indices are set to the
specified value. On the other hand, on records containing index
variables, no parameters must be specified that are not arrays of that index. 

Index variable arrays are used for arrays depending on some standard indices
like the region or atom index, and for multi-dimensional arrays. 

As a convenient feature of NAMELIST input, only those parameters need to be
specified whose values deviate from the predefined default values (given by the
program before reading the NAMELIST record). A few parameters are obligatory,
IMSIL will require them to be specified. Examples of actual input files are
given in Chapter \ref{k:ex}.

